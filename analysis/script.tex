\documentclass{book}
\usepackage{amsmath,amsthm}
\swapnumbers
\usepackage{script} 
\title{Real Analysis, Fall 2017}
\begin{document}


\maketitle

%\setcounter{chapter}{-1}
%\chapter{Preliminaries}


\chapter{The real number line} \label{Ch:real}
\section{Ordered Sets} \label{S:ordered}
One basic property of many number systems (natural numbers, integers, rationals, etc) is that they are {\em ordered}, 
so we say that ``3 is greater than 2'', and so on. 
\begin{defn}
A {\em total order} on a set $S$ is a relation\footnote{A relation is a
comparision operation between two elements which evaluates to either {\em true}
or {\em false}.} $\leq$ satisfying the following axioms:
\begin{enumerate}
[{\quad \bfseries (O1)}]
\item (Reflexivity) For every element $a$, it always holds that $a \leq a$. 
\label{I:order_ref}
\item (Antisymmetry) If $a \leq b$ and $b \leq a$, then it must be that $a = b$.
\label{I:order_anti}
\item (Transitivity) If $a \leq b$ and $b \leq c$, then it holds that $a \leq c$. 
\label{I:order_trans}
\item (Totality) For every pair of elements $a$ and $b$, either $a \leq b$ or $b \leq a$. 
\label{I:order_tot}
\end{enumerate}
We say $S$ is an {\em ordered set}.
\label{D:order}
\end{defn}

\begin{ex}
Find some examples of ordered sets.
\label{Ex:ordered_ex}
\end{ex}

\begin{ex}
Find an example of a {\em partially ordered} set---a set with a relation satisfying axioms (O\ref{I:order_ref})--(O\ref{I:order_trans}) but not (O\ref{I:order_tot}).
\label{Ex:part_ordered_ex}
\end{ex}

\begin{prob}
Suppose $S$ is an ordered set. Formulate a reasonable definition of strict
inequality ($a < b$) in terms of the order relation $\leq$.  Then write
down a definition equivalent to Definition~\ref{D:order} using strict
inequality as the primitive relation; that is, write down a set of axioms that $<$
should satisfy, in terms of which $\leq$ (suitably defined in terms of $<$) has properties (O\ref{I:order_ref})--(O\ref{I:order_tot}).
\label{Pr:equiv_order}
\end{prob}

\begin{defn}
Let $S$ be an ordered set, and $A \subseteq S$ a subset. An {\em upper bound} for $A$ is an element $u \in S$
such that $a \leq u$ for every $a \in A$. If such an element exists, we say $A$ is {\em bounded above}. 

Similarly, a {\em lower bound} for $A$ is an element $l \in S$ such that $l \leq a$ for every $a \in A$. If such a lower bound
exists, we say $A$ is {\em bounded below}. 
\label{D:bounded_above_below}
\end{defn}

\begin{defn}
A {\em least upper bound} or {\em supremum} of a bounded above set $A$ is an element $u_0$ of $S$ such that
\begin{enumerate}
[{\quad \normalfont (i)}]
\item 
$u_0$ is an upper bound for $A$, and
\item
$u_0 \leq u$ for every other upper bound $u$.
\end{enumerate}
We denote a supremum for $A$ (if it exists) by $\sup A$.

Similarly, a {\em greatest lower bound} or {\em infimum} of a bounded below set $A$ is an element $b_0$ of $S$ such that
\begin{enumerate}
[{\quad \normalfont (i)}]
\item 
$b_0$ is a lower bound for $A$, and
\item
$b_0 \geq b$ for every other lower bound $b$.
\end{enumerate}
We denote an infimum for $A$ (if it exists) by $\inf A$.
\label{D:inf_sup}
\end{defn}

\begin{prop}
If a supremum (or infimum) of $A$ exists, then it is unique.
\label{P:unique_sup}
\end{prop}

\begin{prop}
If $A$ and $B$ are subsets of an ordered set $S$ which both have a supremum and an infimum and satisfy $A \subseteq B$, then
\begin{equation}
	\inf B \leq \inf A \leq \sup A \leq \sup B.
	\label{E:sup_inf_ineq}
\end{equation}
\label{P:inf_sup}
\end{prop}

\begin{ex}
Let $\bbZ = \set{\ldots, -2, -1, 0, 1, 2, \ldots}$ denote the set of integers,
with the usual order. Find some examples of subsets $A$ of $\bbZ$ such that
\begin{enumerate}
[{\quad \normalfont (i)}]
\item $A$ is bounded above and below.
\item $A$ is bounded above but not below.
\item $A$ is not bounded above and not bounded below.
\end{enumerate}
Which of these sets have a supremum? Which have an infimum?
\label{Ex:sup_inf_bdd_Z}
\end{ex}

\begin{ex}
Repeat Example \ref{Ex:sup_inf_bdd_Z} with the set $\bbQ$ of rational numbers in place of $\bbZ$. The following Lemma may be of use.
\label{Ex:sup_inf_bdd_Q}
\end{ex}

\begin{lem}
There exists no $q \in \bbQ$ such that $q^2 = 2$. 
\label{L:sqrt_2_irrational}
\end{lem}
\begin{proof}[Proof hint:]
Write $q = \frac a b$ in lowest terms and consider the evenness/oddness of $a$ and $b$.
\end{proof}


\begin{defn}
An ordered set $S$ has the {\em least upper bound property} if every subset which is bounded
above has a supremum. Likewise $S$ has the {\em greatest lower bound property} if every subset which is bounded
below has an infimum.
\label{D:lub_prop}
\end{defn}

\begin{ex}
Does $\bbZ$ have the least upper bound property? Does $\bbQ$? Justify your answers with a proof or counterexample.
\label{Ex:Q_Z_lub}
\end{ex}

\begin{thm}
If $S$ has the least upper bound property, then it has the greatest lower bound property.
\label{T:lub_implies_glb}
\end{thm}


\section{Fields and ordered fields} \label{S:fields}
Of course the familiar number systems have additional structure. Besides the order, we have 
addition, subtraction, multiplication and division.

\begin{defn}
A {\em field} is a set $\bbF$ with two binary operations\footnote{A binary operation is a function/operation taking in two elements of $\bbF$ and returning
a third element of $\bbF$.} 
$+$ and $\cdot$,
called {\em addition} and {\em multiplication}, respectively, satisfying the following axioms:
\begin{enumerate}
[{\quad \bfseries (F1)}]
\item (Associativity of addition) $(a + b) + c = a + (b + c)$ for all $a,b,c$ in $\bbF$.
\label{I:field_ass_add}
\item (Additive identity) There exists an element $0 \in \bbF$ such that $0 + a = a + 0 = a$ for all $a$.
\label{I:field_add_id}
\item (Additive inverses) For each $a$ in $\bbF$ there exists an element $-a$ such that $(-a) + a = a + (-a) = 0$. 
\label{I:field_add_inv}
\item (Commutativity of addition) $a + b = b + a$ for all $a,b$ in $\bbF$.
\label{I:field_comm_add}
\item (Associativity of multiplication) $(a \cdot b) \cdot c = a \cdot (b \cdot c)$ for all $a,b,c$ in $\bbF$.
\label{I:field_ass_mul}
\item (Multiplicative identity) There exists an element $1 \in \bbF$ such that $1 \cdot a = a \cdot 1 = a$ for all $a$.
\label{I:field_mul_id}
\item (Multiplicative inverses) For all $a \neq 0$, there exists an element $a^\inv$ in $\bbF$ such that $a\cdot a^\inv = a^\inv \cdot a = 1$.
\label{I:field_mul_inv}
\item (Commutativity of multiplication) $a\cdot b = b \cdot a$ for all $a,b$ in $\bbF$.
\label{I:field_comm_mul}
\item (Distributivity) $a\cdot (b + c) = a\cdot b + a \cdot c$.
\label{I:field_dist}
\item (Nontriviality) $0 \neq 1$. 
\label{I:field_nondegen}
\end{enumerate}
It is customary to omit the $\cdot$ when writing multiplication; in other words, we usually just write $ab$ instead of $a\cdot b$. Additionally,
we usually denote $a + (-b)$ simply by $a - b$,
and we may also use the notation $\frac 1 a$ in place of $a^\inv$.
It is important to note that subtraction $-$ and division $\frac \cdot \cdot$ are not really distinct operations; they are just syntactic shorthand for addition (resp.\ multiplication) by an additive (resp.\ multiplicative) inverse. 

We also use the usual shorthand $a^n$ in place of $\underbrace{a\cdots a}_\text{$n$ times}$ and $na$ in place of
$\underbrace{a + \cdots + a}_\text{$n$ times}$.
\label{D:field}
\end{defn}

\begin{rmk}
Though we shall be entirely concerned with fields in this course,
you may be familiar with various mathematical objects satisfying fewer 
of the above axioms. A set with a single operation satisfying axioms
(F\ref{I:field_ass_add})--(F\ref{I:field_add_inv}) is a {\em group} which is said
to be {\em commutative} or {\em abelian} if (F\ref{I:field_comm_add}) also holds.

A {\em ring} is a set with two operations satisfying all of the above
except (F\ref{I:field_mul_inv}), (F\ref{I:field_comm_mul}) and (F\ref{I:field_nondegen}). A {\em commutative ring} satisfies (F\ref{I:field_comm_mul}).
According to some conventions, a ring need not satisfy (F\ref{I:field_mul_id}), though such ``rings
without identity'' are sometimes cutely referred to as `{\em rng}'s. If (F\ref{I:field_mul_inv})
holds but not (F\ref{I:field_comm_mul}), then $\bbF$ is called a {\em division
ring}. 

Axiom (F\ref{I:field_nondegen}) might be considered optional for fields, but if we allow $0 = 1$ then $\bbF$
must be the one element set $\set 0$ (you can prove this after you prove
Proposition~\ref{P:field_props_II} below), which for various reasons is best
not regarded as a field.
\end{rmk}

\begin{ex}
Come up with some examples of fields, some with infinitely many and some with
finitely many elements. Can you construct a field with exactly two elements? Three?
\label{Ex:field_exs}
\end{ex}

\begin{prop}
The following properties of addition and multiplication hold in any field. (That is, they follow from the axioms above.)
\begin{enumerate}
[{\quad\normalfont (i)}]
\item (Uniqueness of identities) If an element $b$ in $\bbF$ satisfies $b + a = a$ for some $a$, then $b = 0$. Likewise
if $b$ satisfies $ba = a$ for some $a \neq 0$, then $b = 1$. 
\item (Uniqueness of inverses) If $b$ satisfies $a + b = 0$, then $b = -a$. Likewise, if $b$ satisfies $ba = 1$ then $b = a^\inv$. 
\item (Cancellation) If $a + c = b + c$ then $a = b$. Likewise if $c \neq 0$ and $ac = bc$, then $a = b$.
\item (Inverse of an inverse) $-(-a) = a$ and $(a^\inv)^\inv = a$.
\end{enumerate}
\label{P:field_props_I}
\end{prop}

\begin{prop}
In any field, the following properties hold.
\begin{enumerate}
[{\quad\normalfont (i)}]
\item $0 a = 0$ for all $a$. 
\item If $ab = 0$, then either $a = 0$ or $b = 0$. (We say $\bbF$ ``has no divisors of zero''.)
\label{I:field_props_II_zerodivs}
\item $(-a)b = a(-b) = - (ab)$ for all $a$ and $b$. In particular $-a = (-1)a$. 
\item $(-a)(-b) = ab$ for all $a$ and $b$.
\end{enumerate}
\label{P:field_props_II}
\end{prop}

\begin{prob}
In a field, show that if $b \neq 0$ and $d \neq 0$ then
\[
	\frac a b + \frac c d = \frac{ad + bc}{bd}.
\]
\label{Pr:add_fractions}
\end{prob}

\begin{defn}
An {\em ordered field} is a field $\bbF$ equipped with a total order, so a set
with a relation $\leq$ and two operations $+$ and $\cdot$ 
satisfying axioms (O\ref{I:order_ref})--(O\ref{I:order_tot}) and
(F\ref{I:field_ass_add})--(F\ref{I:field_nondegen}), which is additionally
required to satisfy the following axioms:
\begin{enumerate}
[{\quad \bfseries (OF1)}]
\item (Compatibility of order and addition) If $a \leq b$ then $a + c \leq b + c$ for any $c$.
\label{I:of_add}
\item (Compatibility of order and multiplication) If $a \leq b$ and $0 \leq c$, then $ac \leq bc$.
\label{I:of_mult}
\end{enumerate}
\label{D:ordered_field}
\end{defn}

\begin{ex}
Which examples from Example~\ref{Ex:field_exs} are ordered fields? In case
there is not an obvious order, is there any order at all satisfying (OF\ref{I:of_add}) and (OF\ref{I:of_mult})?
\label{Ex:of_exs}
\end{ex}

\begin{prop}
The following properties always hold in an ordered field.
\begin{enumerate}
[{\quad \normalfont (i)}]
\item If $0 \leq a$ then $-a \leq 0$. 
\item 
If $0 \leq a $ and $0 \leq b$ then $0 \leq ab$. (In fact, this is equivalent to (OF\ref{I:of_mult}) and is often used
in place of it as the other ordered field axiom).
\item
If $a \leq 0$ and $0 \leq b$, then $ab \leq 0$. 
\item
$0 \leq a^2$ for any $a$. In particular $0 < 1$. 
\item 
If $0 < a \leq b$ then $0 < b^\inv \leq a^\inv$. 
\end{enumerate}
In light of Proposition~\ref{P:field_props_II}.\eqref{I:field_props_II_zerodivs} the above identities hold with strict
inequality $<$ used in place of inequality $\leq$. 
\label{P:of_props}
\end{prop}

\begin{prob}
Let $\bbF$ be an ordered field and consider the subset $Z \subset \bbF$ generated by taking $0$, $1$, $1 + 1$, $1 + 1 + 1$, etc.\ along with $-1$, $-1 - 1$, $-1 -1 -1$, etc. Show that
this set is in bijection with the set of integers $\bbZ$. 

Likewise, let $Q \subset \bbF$ be the subset generated by taking the multiplicative inverses of the nonzero elements in $Z$ along with their integer multiples. Show that this set is in bijection
with $\bbQ$. 

Thus every ordered field contains a copy of $\bbQ$, which may be regarded as the ``smallest'' possible ordered field.
\label{Pr:Q_subfield}
\end{prob}

\begin{defn}
Let $\bbF$ be an ordered field. The {\em absolute value} or {\em magnitude} of a number $a \in \bbF$ is defined by
\[
	\abs a = \begin{cases} a & \text{if $a \geq 0$}, \\ -a & \text{if $a < 0$.} \end{cases}
\]
\label{D:abs_value}
\end{defn}

\begin{propstar}
The absolute value satisfies the following properties. For all $a$ and $b$ in $\bbF$:
\begin{enumerate}
[{\quad \normalfont (i)}]
\item $\abs a \geq 0$.
\item $\abs a = 0$ if and only if $a = 0$.
\item $\abs{ab} = \abs a \abs b$.  
\item (Triangle inequality) $\abs{ a + b} \leq \abs a + \abs b$.
\label{I:abs_triang}
\item (Reverse triangle inequality) $\babs{\abs a - \abs b} \leq \abs{a - b}$.
\label{I:abs_rev_triang}
\end{enumerate}
\label{P:abs_props}
\end{propstar}

\begin{rmk}
Combining \eqref{I:abs_triang} and \eqref{I:abs_rev_triang} of the last proposition gives the useful strings of inequalities:
\begin{equation}
	\abs a - \abs b \leq \babs{ \abs a - \abs b} \leq \abs{a + b} \leq \abs a + \abs b, \quad \text{and}
	\quad \abs a - \abs b \leq \babs{ \abs a - \abs b} \leq \abs{a - b} \leq \abs a + \abs b.
	\label{E:full_triangle}
\end{equation}
\end{rmk}

\begin{defn}
The {\em distance} between numbers $a$ and $b$ in an ordered field $\bbF$ is the quantity
\[
	d(a,b) = \abs{a - b}.
\]
\label{D:distance_of}
\end{defn}

\begin{propstar}
The distance satisfies the following properties. For all $a$, $b$, and $c$ in $\bbF$:
\begin{enumerate}
[{\quad \normalfont (i)}]
\item $d(a,b) \geq 0$.
\item $d(a,b) = 0$ if and only if $a = b$. 
\item (Symmetry) $d(a,b) = d(b,a)$.
\item (Triangle inequality) $d(a,c) \leq d(a,b) + d(b,c)$.
\end{enumerate}
\label{P:distance_props_of}
\end{propstar}

\begin{lem}[Suprema/infima in an ordered field]
Let $A$ be a bounded above subset of an ordered field. Then $u = \sup A$ if and only if 
\begin{enumerate}
[{\quad \normalfont (i)}]
\item  
$a \leq u$ for all $a \in A$ (i.e., $u$ is an upper bound), and
\item
for every $\ve > 0$, there exists $a \in A$ such that $u - \ve < a$ (i.e., $u - \ve$ fails to be an upper bound).
\end{enumerate}
Similarly, if $A$ is bounded below, then $b = \inf A$ if and only if
\begin{enumerate}
[{\quad \normalfont (i)}]
\item  
$b \leq a$ for all $a \in A$, and 
\item
for every $\ve > 0$, there exists $a \in A$ such that $a < b + \ve$.
\end{enumerate}
\label{L:sup_inf_characterization}
\end{lem}

\begin{defn}[$\pm \infty$ notation]
As a notation convention, it is useful to introduce the symbols $+\infty$ and
$-\infty$ when speaking of suprema and infima in an ordered field. We write
$\sup A = + \infty$ if $A$ is not bounded above, and $\inf A = -\infty$ if $A$
is not bounded below.  With these conventions $\sup A$ and $\inf A$ are always
defined for a nonempty set $A$, and \eqref{E:sup_inf_ineq} holds identically whenever $A \subseteq B$.

A more formal way to do this is to embed $\bbF$ into a larger ordered set $\ol \bbF = \bbF \cup \set{+\infty,-\infty}$ with the order
defined so that $-\infty < a < + \infty$ for all $a \in \bbF$. Note that $\ol \bbF$ is {\em not} a field, though we may observe the following notation conventions: if
$a > 0\in \bbF$, then
\[
\begin{aligned}
	a + (+\infty) &= + \infty, &a + (-\infty) &= -\infty,
	& a(+\infty) &= +\infty, &a(-\infty) = -\infty,
	\\ (-a)(+\infty) &= -\infty, &(-a)(-\infty) &= +\infty,
	& \frac{\pm a}{\pm\infty} &= 0.
\end{aligned}
\]
Expressions such as $+\infty - \infty$ and $\pm\infty/\pm \infty$ are not defined.
\label{D:pm_infty}
\end{defn}


\section{Completeness and the real number field} \label{S:field_complete}
\begin{defn}
An ordered field $\bbF$ is {\em complete} if it satisfies the least upper bound property (c.f.\ Definition~\ref{D:lub_prop}), in other words,
if for every bounded above subset $A \subset \bbF$, the supremum (least upper bound) $\sup A$ exists in $\bbF$.
\label{D:complete_field}
\end{defn}


\begin{thmdag}[Characterization/definition of $\bbR$]
There exists a unique\footnote{Here ``uniqueness'' means the following: given two complete ordered fields $F_1$ and $F_2$, there exists an {\em isomorphism} (a bijection compatible with the order and field operations) $\phi : F_1 \to F_2$. Moreover $\phi$ is unique. Using $\phi$ we can regard $F_1$ and $F_2$ as being ``the same'' field.} complete ordered field called the {\em real numbers} and denoted by $\bbR$.
\label{T:reals}
\end{thmdag}

\begin{rmk}
We omit the proof of Theorem~\ref{T:reals} for now; we may come back to it
later on. However, it is worth mentioning one construction which is possible at
this point: define a {\em Dedekind cut} to be a subset $A \subset \bbQ$ of the rationals
with the properties that 
\begin{enumerate}
[{\quad \normalfont (i)}]
\item $A$ is neither empty nor all of $\bbQ$,
\item if $q \in A$ and $p < q$, then $p \in A$,
\item if $q \in A$ then $q < r$ for some $r \in A$.
\end{enumerate}
In other words, a cut is essentially a half infinite open interval in $\bbQ$;
take as an example $\set{q \in \bbQ : q < 2} = (-\infty, 2)$. It is 
tempting to want to write $\set{q \in \bbQ : q < \sqrt 2}$ as another example, but this
is ill-specified since we do not have such a number as $\sqrt 2$ at this point.
The equivalent set may be specified as $\set{q \in \bbQ : q < 0 \text{ or }
q^2 < 2}$. The idea here is that real numbers are represented by the ``upper endpoints'' of the cuts,
though since these are not well-defined, the whole cut stands in as a replacement.

It is then possible to define an order, addition, and multiplication on the set
of Dedekind cuts (order and addition are straightforward; multiplication is a
little tricky) and verify that they satisfy all the axioms of an ordered field
along with completeness, with subfield $\bbQ$ identified with those cuts of the
form $\set{q \in \bbQ : q < p}$ for $p \in \bbQ$.
\end{rmk}

\begin{defn}
An ordered field $\bbF$ is {\em Archimedean} if for every $a \in \bbF$, there exists
an integer\footnote{Here we are identifying a subset of $\bbF$ with the integers
as in Problem~\ref{Pr:Q_subfield}.} $N$ such that $a \leq N$.
\label{D:archimedean}
\end{defn}

\begin{exstar}
Show that $\bbQ$ is Archimedean.
\label{Ex:Q_Archimedean}
\end{exstar}

\begin{ex}[Research Allowed]
Find an example of a non-Archimedean field.
\label{Ex:non_Archimedean_field}
\end{ex}

\begin{thm}
As a complete ordered field, $\bbR$ is Archimedean.
\label{T:R_archimdedean}
\end{thm}

\begin{prop}
A field is Archimedean if and only if, for every $a > 0$, there exists a positive integer $N$ such that 
\[
	0 < \frac 1 N < a.
\]
\label{P:archimedean_small}
\end{prop}

\begin{rmk}
The Archimedean property says that a field has no ``infinitely large'' elements, and via Proposition~\ref{P:archimedean_small}, it implies
that there are no ``infinitely small'' elements. The next result gives a technically useful if strange seeming characterization of the zero element.
\end{rmk}

\begin{cor}
In an Archimedean field, if $0 \leq a$ and $a < \ve$ for every $0 < \ve$, then $a = 0$.
\label{C:zero_small}
\end{cor}

\begin{thm}[Density of $\bbQ$ in $\bbR$]
Let $a$ and $b$ be real numbers with $a < b$. Then there exists a rational number $q$ such that
\[
	a < q < b.
\]
We say $\bbQ$ is {\em dense} in $\bbR$.
\label{T:Q_dense}
\end{thm}

\begin{rmk}
This may be a surprising result, especially when juxtaposed with the following one.
Recall that an infinite set is said to be {\em countable} if it is in bijection with the set $\bbN = \set{1,2,3,\ldots}$ of natural numbers. 
\end{rmk}

\begin{thm}\mbox{}
\begin{enumerate}
[{\quad \normalfont (i)}]
\item $\bbQ$ is countable.
\item $\bbR$ is uncountable. 
\end{enumerate}
\label{T:countable_uncoutable}
\end{thm}

One more result at this point will be useful later on, though the proof is rather technical and tricky, so you may 
go ahead and take it as given rather than trying to prove it.

\begin{thmdag}[Positive $n$th roots]
For every $y> 0$ in $\bbR$ and $n \in \bbN$, there exists a unique $x > 0$ in $\bbR$ such that $x^n = y$.
\label{T:nthroots}
\end{thmdag}
\noindent The proof is obtained from the following two results, the first of which is more or less straightforward while
the second is the tricky one.
\begin{lem}
For fixed $y > 0$ and $n \in \bbN$, the set $E = \set{t \in \bbR : 0 < t,\ t^n < y}$ is nonempty and bounded above.
\label{L:nthroots1}
\end{lem}
\begin{lemdag}
The element $x = \sup E$ satisfies $x^n = y$.
\label{L:nthroots2}
\end{lemdag}

\section{Sequences of real numbers} \label{S:seqs}
While least upper bounds give an expedient way to express the
completeness of $\bbR$, {\em sequences} play a much more ubiquitous role in
analysis.

\begin{defn}
A {\em sequence} of real numbers is a function\footnote{Recall that a {\em function} $f : A \to B$ is an assignment to every element $a$ of the {\em domain} $A$ an element $b = f(a)$ of the {\em target} $B$.} from $\bbN$ into $\bbR$. As a
matter of notation, if $x : \bbN \to \bbR$ is a sequence, we prefer to write $x_n$ instead of $x(n)$, and
denote the sequence by
\[
	(x_1,x_2,x_3,\ldots), 
	\quad \text{or} \quad  
	(x_n)_{n=1}^\infty, 
	\quad \text{or just} \quad 
	(x_n).
\]
It is permissible and often convenient
to index a sequence starting from $0$ instead of $1$, or
starting from a number greater than $1$. 
\label{D:real_sequence}
\end{defn}

\begin{defn}
A sequence $(x_n)$ in $\bbR$ is said to be 
\begin{enumerate}
[{\quad \normalfont (i)}]
\item {\em bounded} if there exists some $B > 0$ such that $\abs {x_n} \leq B$ for all $n$.
\item {\em increasing} if $x_n \leq x_{n+1}$ for all $n$. It is {\em strictly increasing} if $x_n < x_{n+1}$ for all $n$. 
\item {\em decreasing} if $x_n \geq x_{n+1}$ for all $n$. It is {\em strictly decreasing} if $x_n > x_{n+1}$ for all $n$. 
\item {\em monotone} if it is either increasing or decreasing.
\item {\em convergent} if there exists some $L \in \bbR$ with the following property: for every $\ve > 0$ there exists $N \in \bbN$ such that
\[
	\text{for all $n \geq N$,} 
	\quad 
	\abs{x_n - L} < \ve.
\]
Equivalently, for every $\ve > 0$, the interval $(L - \ve, L + \ve) \subset \bbR$ contains all but finitely many terms of the sequence.
In this case we say $L$ is the {\em limit} of the sequence $(x_n)$ and write 
$\lim_{n\smallto\infty} x_n = L$ or $x_n \smallto L$.
\end{enumerate}
\label{D:seq_properties}
\end{defn}

\begin{expn}
Explain in plain English what is meant by the limit of a sequence. Address the order of the quantifiers (i.e., ``for all'' or ``there exists''): if the order or type of the quantifiers is changed, why is this a bad definition of limit?
\label{Expn:limit}
\end{expn}

\begin{prop}
The limit of a sequence, if it exists, is unique.
\label{P:limit_unique}
\end{prop}


\begin{exstar}
\mbox{}
\begin{enumerate}
[{\quad \normalfont (i)}]
\item Show that the constant sequence $x_n = c$ for all $n$ converges and $\lim x_n = c$.
\item Show that $x_n = \frac{1 + 3n}{1 + 5n}$ has limit $\frac 3 5$.
\end{enumerate}
\label{Ex:1_over_n}
\end{exstar}

\begin{ex}
\mbox{}
\begin{enumerate}
[{\quad \normalfont (i)}]
\item Show that $x_n = \frac 1 n \smallto 0$. 
\item If $0 \leq p < 1$, show that $x_n = p^n \smallto 0$. [Hint: such $p$ can be written as $p = \frac 1{1 + a}$ for $a > 0$. The binomial estimate $(1 + a)^n \geq 1 + na$ for $a \geq 0$ is also useful here.]
\end{enumerate}
\label{Ex:seqs}
\end{ex}

\begin{prop}
If a sequence converges, then it is bounded.
\label{P:cnvgt_seq_bdd}
\end{prop}


\begin{thm}
In $\bbR$, every bounded monotone sequence converges.
\label{T:MSP}
\end{thm}

\begin{rmk}
The previous property of $\bbR$ is referred to as the {\em monotone sequence
property}. In fact, it is equivalent to completeness: it can be proved that an
ordered field in which every bounded monotone sequence converges has the least upper bound property.
\end{rmk}

\begin{prob}
Let $x_n = \sqrt{n^2 + 1} - n$. Show $(x_n)$ converges and compute its limit.
\label{Pr:limit_renormalized}
\end{prob}


\begin{thm}
Let $(x_n)$ and $(y_n)$ be convergent sequences with limits $x$ and $y$, respectively. Then
%$(x_n + y_n)$, $(x_ny_n)$, $(-x_n)$ and (if $x > 0$ and $x_n > 0$ for all $n$) $(x_n^\inv)$ are convergent sequences and
\begin{enumerate}
[{\quad \normalfont (i)}]
\item $x_n + y_n \smallto x+y$.
\item $-x_n \smallto -x$. 
\item $x_ny_n \smallto xy$.
\item If $x_n \neq 0$ for all $n$ and $x \neq 0$, then $x_n^\inv \smallto x^\inv$.
\item If $x_n \leq y_n$ for all $n$, then $x \leq y$. 
\label{I:limit_thms_ineq}
\end{enumerate}
\label{T:limit_thms_seqs}
\end{thm}

\begin{rmk}
Combining the above, it follows that $x_n - y_n \smallto x - y$ and $x_n/y_n \smallto x/y$ (provided $y_n \neq 0$ and $y \neq 0$).
\end{rmk}

\begin{prob}
Let $p(t) = a_0 + \cdots + a_l t^l$ and $q(t) = b_0 + \cdots + b_m t^m$ be
polynomials with real coefficients, where $a_l \neq 0$ and $b_m \neq 0$. If $l \leq m$, prove that
\[
	\lim_{n\smallto \infty} \frac{p(n)}{q(n)} = 
	\begin{cases} 0, & \text{if $l < m$, and} 
	\\ a_l/b_m, & \text{if $l = m$.} 
	\end{cases}
\]
\label{Pr:rational_limit}
\end{prob}

\begin{prob}
Define a sequence inductively by setting $x_1 = \sqrt 2$ and for $n \geq 2$, $x_n = \sqrt{2 + x_{n-1}}$. Prove 
that $(x_n)$ converges and find its limit.
\label{Pr:inductive_sequence}
\end{prob}

\begin{exstar}
Show that strict inequality cannot be obtained in Theorem~\ref{T:limit_thms_seqs}.\eqref{I:limit_thms_ineq}, by producing an example where $x_n < y_n$ for all $n$ but $x = y$.
\label{Ex:strict_limit_ineq}
\end{exstar}

\begin{exstar}[Harmonic series/sequence]
Define a sequence by $x_1 = 1$, $x_2 = 1 + \frac 1 2$, and $x_n = 1 + \cdots + \frac 1 n$. Show $(x_n)$ is monotone increasing, but unbounded above, hence
does not converge. [Possible hint: compare to the sequence $y_1 = 1$, $y_2 = 1 + \frac 1 2$, $y_3 = 1 + \frac 1 2 + \frac 1 4$, $y_4 = 1 + \frac 1 2 + \frac 1 4 + \frac 1 4$,
$y_5 = 1 + \frac 1 2 + \frac 1 4 + \frac 1 4 + \frac 1 8$
etc., where $y_n$ has each term from the corresponding $x_n$ replaced by the largest power of $2^\inv$ which is less than or equal to it.]
\label{Ex:harmonic}
\end{exstar}

The following limits are useful to know but tricky to prove at this
point\footnote{In fact they become quite easy to prove once we have developed
the logarithm, but we do not have this yet.}, so you can take them as given
rather than trying to prove them.
\begin{propdag}
\mbox{}
\begin{enumerate}
[{\quad \normalfont (i)}]
\item For any $a > 0$, the sequence $a^{1/n} \smallto 1$.
\item The sequence $n^{1/n} \smallto 1$. 
\end{enumerate}
\label{P:nthrootseq}
\end{propdag}

In addition to bounded monotone sequences, another very useful class of sequences which ``ought to converge'' are the {\em Cauchy sequences}.
\begin{defn}
A sequence $(x_n)$ is {\em Cauchy} if for all $\ve > 0$, there exists $N \in \bbN$ such that
\begin{equation}
	\text{for every $n, m \geq N$,}
	\quad
	\abs{x_n - x_m} < \ve.
	\label{E:cauchy}
\end{equation}
\label{D:cauchy}
\end{defn}
\begin{rmk}
Intuitively, a Cauchy sequence is one in which the {\em tails} of the sequence (i.e., the sequences $(x_n)_{n=N}^\infty$ for various $N$) become arbitrarily ``bunched up''.
\end{rmk}

\begin{prop}
Every Cauchy sequence is bounded.
\label{P:cauchy_bndd}
\end{prop}

\begin{prop}
Every convergent sequence is Cauchy.
\label{P:convergent_cauchy}
\end{prop}

The converse is not true in general; however one of the distiguishing features of $\bbR$ over $\bbQ$ as a complete
ordered field is the following main result.

\begin{thm}
In $\bbR$, every Cauchy sequence converges.
\label{T:R_cauchy_complete}
\end{thm}
\begin{proof}[Proof hint]
For each $k$ let $a_k = \sup \set{x_n : n \geq k}$. Then $(a_k)$ is a bounded decreasing sequence. Show that $x_n \smallto a$, 
where $a = \lim_k a_k$.
\end{proof}

\begin{rmk}
This property is known as the {\em Cauchy completeness} of $\bbR$. It is possible to
show that it is equivalent to both the least upper bound property and to the monotone
sequence property.
\end{rmk}

\begin{ex}
Show that \eqref{E:cauchy} cannot be replaced by ``for every $n \geq N$, $\abs{x_n - x_{n+1}} < \ve$'', by finding
an example of a divergent sequence in $\bbR$ with the latter property. 
\label{Ex:divergent_almost_cauchy}
\end{ex}

\begin{rmk}
The previous exercise shows that it is generally not enough for pairs of
adjacent elements in the sequence to be getting close together; rather, we need the distance between
any pair of not-necessarily-adjacent elements in some tail of the sequence to be getting close.
%
On the other hand, if adjacent pairs become close fast enough, then the sequence actually is Cauchy, as the next
result shows.
\end{rmk}

\begin{lem}
Let $(x_n)$ be a sequence in $\bbR$ such that for all $n \in \bbN$,
\[
	\abs{ x_n - x_{n+1}} \leq \frac a {2^n},
	\quad \text{for some $a > 0$.}
\]
Then $(x_n)$ is Cauchy, and therefore convergent. (In fact $a/2^n$ can be replaced by $a/b^n$ for any $b > 1$.)
\label{L:suff_cauchy}
\end{lem}
%\begin{proof}[Proof hint:] 
%Use the convergence of the geometric series $\sum_{n=1}^\infty 2^{-n} = 1$.
%\end{proof}

%% Proof requires geometric series

\begin{rmk}
Having discussed sequences, we can now mention two more methods of constructing $\bbR$ from $\bbQ$. In both cases we consider as elements 
equivalence classes of sequences $(x_n)$ in $\bbQ$, with the equivalence relation $(x_n) \sim (y_n)$ if $x_n - y_n \smallto 0$.

The first method uses equivalence classes of {\em bounded
increasing sequences}, while the second method uses equivalence classes
of {\em Cauchy sequences}. Each element $q \in \bbQ$ is represented by the
equivalence class of the constant sequence $x_n = q$ for all $n$. 

In either method, one has to define a notion of addition, multiplication, and order on
sequences, show these are well-defined on equivalence classes, and prove that
the set of equivalence classes satisfies all the axioms for an ordered field,
along with some version of completeness (usually the monotone sequence property
if you are using monotone sequences, and the Cauchy sequence property if you
are using Cauchy sequences).
\end{rmk}

\begin{thm}[Squeeze theorem]
Let $(x_n)$, $(y_n)$ and $(z_n)$ be sequences in $\bbR$ such that $x_n \leq y_n \leq z_n$ for all $n$, and
suppose that $x_n \smallto l$ and $z_n \smallto l$. Then $(y_n)$ also converges and $\lim_n y_n = l$.
\label{T:squeeze}
\end{thm}

\begin{expn}
Explain in plain English what is the significance of the completeness of $\bbR$. Why is this a useful property to have?
\label{Expn:completeness}
\end{expn}

\section{Subsequences} \label{S:subseqs}
\begin{defn}
Let $(x_n)$ be a sequence in $\bbR$. A {\em subsequence} of $(x_n)$ is a
sequence $(x_{n_k})_{k=1}^\infty$ where $n_1 < n_2 < n_3 <\cdots$ form a strictly
increasing sequence $(n_k)$ of natural numbers.

If $x_{n_k} \smallto l$ for some subsequence $(x_{n_k})$ of $(x_n)$, we say $l$ is a {\em subsequential limit} of
$(x_n)$. 
\label{D:subseq}
\end{defn}

\begin{ex}
Find examples of a non-convergent sequence $(x_n)$ with
\begin{enumerate}
[{\quad \normalfont (i)}]
\item No subsequential limits.
\item Exactly one subsequential limit.
\item Exactly two subsequential limits.
\item Exactly three subsequential limits.
\item Infinitely many subsequential limits.
\end{enumerate}
\label{X:subseq_limits}
\end{ex}

\begin{prop}
A number $l \in \bbR$ is a subsequential limit of a sequence $(x_n)$ if and only if, for
every $\ve > 0$, 
\[
	\abs{x_n - l} < \ve
\]
for infinitely many $n \in \bbN$.
\label{P:subseq_limit}
\end{prop}

\begin{prop}
If $x_n \smallto x$, then every subsequence of $(x_n)$ converges to $x$. 
\label{P:every_subseq_converges}
\end{prop}

\begin{prop}
If $(x_n)$ is a Cauchy sequence (not necessarily in $\bbR$, perhaps in $\bbQ$), and $x_{n_k} \smallto x$ for some subsequence $(x_{n_k})$, then $x_n \smallto x$.
\label{P:cauchy_plus_subseq}
\end{prop}

\begin{thm}
Every sequence in $\bbR$ or $\bbQ$ has a monotone subsequence (either increasing or decreasing).
\label{T:monotone_subseq}
\end{thm}

\begin{cor}
If $(x_n)$ is a sequence in $[a,b] \subset \bbR$, for some $a \leq b$, then
$(x_n)$ has a convergent subsequence in $\bbR$.
\label{C:mini-HB}
\end{cor}


Prior to discussing limit superior and inferior, it is convenient to make the following definition. 
\begin{defn}
Given a sequence, we say $x_n \smallto +\infty$ if, for every $M > 0$, there exists $N \in \bbN$ such that
\[
	M < x_n \quad \text{for all $n \geq N$}.
\]
Likewise, we say $x_n \smallto - \infty$ if for every $M < 0$, there exists $N \in \bbN$ such that
\[
	x_n < M \quad \text{for all $n \geq N$}.
\]
In neither case do we regard $(x_n)$ as a convergent sequence in $\bbR$;
however, it is possible to regard it as convergent in the {\em extended real
numbers} $\ol \bbR = \bbR \cup \set{\pm \infty}$, regarded as an ordered set
(but not a field) as in Definition~\ref{D:pm_infty}.
\label{D:lim_pm_infty}
\end{defn}

\begin{prop}
If $(x_n)$ is a monotone sequence, then $x_n \smallto l$ for some $l \in \bbR \cup \set{\pm \infty}$.
\label{P:monotone_extended_converges}
\end{prop}


\begin{defn}
Let $(x_n)$ be a sequence in $\bbR$ and let $L = \set{ l \in \bbR \cup \set{\pm
\infty} : x_{n_k} \smallto l, \text{ for some subsequence $(x_{n_k})$}}$ be the
set of its subsequential limits, including possibly $+\infty$ and $-\infty$.
The {\em limit superior} of $(x_n)$ is the supremum 
\[
	\limsup x_n = \sup L,
\]
and the {\em limit inferior} of $(x_n)$ is the infimum
\[
	\liminf x_n = \inf L.
\]
\label{D:limsup}
\end{defn}

\begin{prop}
For every sequence $(x_n)$ in $\bbR$, there exists a subsequence $(x_{n_k})$ such that
\[
	x_{n_k} \smallto \limsup x_n.
\]
Likewise, there exists a subsequence $(x_{n_k})$ such that $x_{n_k} \smallto \liminf x_n$.

In other words, in the definition of limit superior, $\limsup x_n = \sup L$ and
$\liminf x_n = \inf L$ are actually elements of $L$.
\label{P:limsup_subseq}
\end{prop}

The next result justifies the names ``limit inferior'' and ``limit superior''.
\begin{prop}
An equivalent characterization of limit superior and limit inferior are as follows. Let $(x_n)$ be a real sequence
and for each $m \in \bbN$, let $a_m = \inf \set{x_n : n \geq m}$ and $b_m = \sup \set{x_n : n \geq m}$. Then
$a = \liminf x_n$ if and only if
\[
	a = \lim a_m = \textstyle \lim_m \inf \set{x_n : n \geq m} = \sup \set{\inf \set{x_n : n \geq m}}.
\]
Likewise, $b = \limsup x_n$ if and only if
\[
	b = \lim b_m = \textstyle \lim_m \sup \set{x_n : n \geq m} = \inf \set{\sup \set{x_n : n \geq m}}.
\]
\label{P:limsup_alt}
\end{prop}


\begin{prop}
For a sequence $(x_n)$,
\[
	\liminf x_n \leq \limsup x_n
\]
with equality if and only if $(x_n)$ converges to this value.
\label{P:limsup_ineq}
\end{prop}

\end{document}
\section{Series} \label{S:series}

\chapter{Metric space topology} \label{Ch:metric}
\end{document}
